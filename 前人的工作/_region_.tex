\message{ !name(main.tex)}\documentclass{amsart}

\usepackage{ctex}
\usepackage{amsmath}
\usepackage{amsthm}
\usepackage{cite}
\usepackage{wyz}


\title{传感器故障诊断和减小}
\author{}
\date{}

\begin{document}

\message{ !name(main.tex) !offset(-3) }


\maketitle
\tableofcontents

\section{前人的工作}

\subsection{故障诊断}
\subsection{故障减小}
\subsection{曝光异常}
文献 \cite{liliming} 本文研究了几种典型机器学习算法在不同图像降质因素作用下的人脸识别性能,进一步分析了上述算法的鲁棒性。
从仿真结果来看,
对于椒盐噪声及运动模糊影响的人脸图像,
PCA算法鲁棒性较好,
但该算法性能显著依赖于图像曝光条件影响。
相比之下,当图像曝光条件不佳时,采用RBFNN作为人脸识别算法具有相对较好的鲁棒性。
文献说说明了图像降质数学模型
\begin{itemize}
\item 椒盐噪声
  实验研究表明,摄像机拍摄图像过程中,图像传感器、传输信道、解码处理等部件或因素将引入椒盐噪声,
  在图像上呈现黑白杂点,进而影响后续图像处理。
  设图像$I(x,y)$为N位图,椒盐噪声密度为$d_{xp}$.
  满足$d_{xp}\in [0,1]$.
  ,则则叠加椒盐噪声后的图像可以表示为:
  \begin{equation}
    \label{eq:01}
  \left\{\begin{array}{l}
g(x, y)=I(x, y) * \gamma(x, y) \\
\operatorname{Prob}(\gamma(x, y)=0)=\frac{d_{s p}}{2} \\
\operatorname{Prob}\left(\gamma(x, y)=\frac{2^{N}-1}{I(x, y)}\right)=\frac{d_{\operatorname{sen}}}{2} \\
\operatorname{Prob}(\gamma(x, y)=1)=1-d_{s p}
\end{array}\right.
  \end{equation}

\item 曝光异常
  同态滤波方法利用光照反射模型,
  把频率过滤和灰度变换结合,
  可以在不损失图像细节的前提下调解图像的光照条件。
  考虑到同态滤波具有上述特点,
  因此本文将其用作曝光异常的图像仿真方法。
  设原始图像为,$I(x,y)$,首先对其取对数并做傅里叶变换,提取高频及低频分量:
  \begin{equation}
    \label{eq:02}
Z(u, v)=F(z(x, y))=F(\ln (I(x, y))).
  \end{equation}

  之后,采用频域滤波函数对图像进行增强处理,
  本文采用高斯型高通滤波器作为滤波函数,处理如式\ref{eq:03}所示。
  其中M,N分别表示图像的行、列像素数。
  $D_0$为截止频率,$c$为锐化系数,$R_h$为高频增益,$R_t$为低频增益。

  \begin{equation}
    \label{eq:03}
\left\{\begin{array}{l}
S(u, v)=H(u, v) \cdot Z(u, v) \\
H(u, v)=\left(R_{h}-R_{l}\right) \cdot \gamma(u, v)+R_{l} \\
\gamma(u, v)=\left(1-\exp \left(-c \cdot\left(\frac{D(u, v)}{D_{0}}\right)^{2}\right)\right) \\
D(u, v)=\left(\left(u-\frac{M}{2}\right)^{2}+\left(v-\frac{N}{2}\right)^{2}\right)^{0.5}
\end{array}\right.
  \end{equation}

  最后,对频域图像做傅里叶反变换并取指数,计
  算过程如式\ref{eq:04}所示。

  \begin{equation}
    \label{eq:04}
g(x, y)=e^{s(x, y)}=e^{F-1}(S(u, v))
  \end{equation}

\item 运动模糊

  在照片曝光期间,相机与被摄物体之间发生相对运动造成的图像模糊称为运动模糊。
  受运动模糊影响的图像往往在视觉上表现为图像像素整体沿某一方向具有拖影效果。
  当像素位移量偏大时,将严重影响图像质量,从而降低人脸识别准确率。
  考虑到变速或非直线运动在一定条件下可以被分解为分段匀速直线运动,因此匀速直线运动造成的运动模糊具有普适的研究意义。
  从物理场景上看,图像发生运动模糊的原因是被摄图像经过一定距离延迟后再进行叠加。将静止条件下的图像表示为$I(x,y)$.
  设快门打开期间,图像传感器与被摄物体保持水平
  匀速直线运动,则图像褪化模型可以表示为示:
  \begin{equation}
    \label{eq:05}
g(x, y)=\frac{1}{L} \sum_{i=0}^{L} f(x-i, y) \Delta t
  \end{equation}
  其中,$L$表示图像发生整体位移的像素长度近似值。
\end{itemize}


\cite{gaohuihuang}
准确判断巡检图片中无效图片因为巡检图片随着列车的行驶不间断地拍摄,因而在列车进站、临时停车等因素造成告诉摄像机摄入大量重复图片,为接触网异常检测增加了较多额外工作量。
除此之外,因为列车走向可能正对太阳造成曝光过度,或者因为夜间行驶造成巡检图片光线不足进而缺少必要的细节信息
。本系统需要有效检测出以上无效图片,标记图片信息并提示系统用户。
随机概率统计过滤无效图片

图像曝光过度或者光线过暗会造成图像丢失细节信息。
为了确保无效图片对绝缘子的识别与异常检测造成干扰,需要在巡检图片中判定
无效图片并做标记便于系统查询。
  \begin{equation}
    \label{eq:06}
\begin{array}{l}
\text { FlagBright }(\mathrm{x}, \mathrm{y})=\left\{\begin{array}{ll}
1, & I(x, y)>215 \\
0, & I(x, y) \leq 215
\end{array}\right. \\
\text { FlagDark }(\mathrm{x}, \mathrm{y})=\left\{\begin{array}{ll}
1, & I(x, y)<30 \\
0, & I(x, y) \geq 30
\end{array}\right.
\end{array}
\end{equation}
图像无效的判别公式如 \ref{eq:07}所示。
\begin{equation}
  \label{eq:07}
\operatorname{ImgErr}=\left\{\begin{array}{cc}
1, & \text { if } \frac{\left(\sum_{i=1}^{\text {width }} \sum_{j=1}^{\text {height }} \operatorname{Flag} \operatorname{Bright}(x, y)\right)}{\text { width * height }}>80 \% \\
\text { 1, } \quad \text { if } \frac{\left(\sum_{i=1}^{\text {width }} \sum_{j=1}^{\text {height }} \operatorname{Flag} \operatorname{Dark}(x, y)\right)}{\text { width * height }}>80 \% & \text { else } \\
0, & \text { when }
\end{array}\right.
\end{equation}

剔除部分重复拍摄照片
(1)通过观察大量巡检图片,取倾斜角在[84~96]之间的直线称为偏竖直直线。
(2)对 ROI 区域内图像进行 Hough 变换提取直线信息,并保留偏竖直直线数
据.
(3)比较直线分布区间内有最多竖直直线数,取区间数值小(即巡检图片中左侧位置)的线段的中心点横坐标标定接触网支撑结构位置 w,当该位置处于原图 ROI 位置宽度 width/6 处左侧时,认定该位置为不重叠区域,其他区域为重叠区域。
图像增强
(1)对比度增强
(2)中值滤波(文章没有提到这种方法)

\subsection{图像处理}
\label{sec:02}

\subsubsection{图像增强}
\label{sec:0201}

文献\cite{许欣}(1)提出了一种快速Retinex图像增强方法,模拟了人类视觉系统的全局和局部自适
应性.
(2)提出了一种改进的自动颜色均衡化方法,用于图像对比度的增强.
(3)提出了一种改进的结合视觉感知特性的变分框架下的彩色图像增强方法.
(4)提出了一种结合视觉特性的梯度域图像增强方法.
(5)针对同一图像采用不同增强方法处理的结果之间可存在互补优缺点的特点,提出了采用梯度域融合的方法改善图像视觉效果的增强方法.


基于核的Retinex及其反对称化
Bena]lmfo等睇1提出了Retinex基于核函数的实现(Kernel Based Retinex.,KBR),其形式为
\begin{equation}
  \label{eq:08}
\begin{aligned}
L(x)=& \sum_{v \in \Omega} w(x, y) f\left(\frac{I(x)}{I(y)}\right) \operatorname{sign}^{+}(I(y)-I(x)) \\
&+\sum_{v \in \Omega} w(x, y) \operatorname{sign}^{-}(I(y)-I(x))
\end{aligned}
\end{equation}

其中,f(·)为对像素值间的比值进行调整的函数,
\begin{equation}
  \label{eq:09}
\operatorname{sign}^{+}(\xi)=\left\{\begin{array}{ll}
1, & \text { if } \xi>0 \\
\frac{1}{2}, & \text { if } \xi=0, \quad \operatorname{sign}^{-}(\xi)=1-\operatorname{sign}^{+}(\xi) \\
0, & \text { if } \xi<0
\end{array}\right.
\end{equation}

核$w(x,y)$为$||x-y||$的函数,具有对称性,且$\sum_{y\in \Omega}(x,y)=1,\forall x \in \Omega$.
此形式的性质满足原有Retinex方法的特点,即可消除彩色图像色偏,可增强欠曝光图像中的细节,只提高像素亮度值等.
在此基础上,他们还对KBR作了“反对称化”,形式


其中,$sign_{0}(·)$为符号函数,此形式改变了原有的各种Retinex:实现方法(若不考虑色调重整的步骤)只能提高图像亮度的特点,既可处理欠曝光图像,也可处理过曝光图像.他们还指出了反对称化的KBR与ACE模型的关系.



\begin{itemize}
\item 1.2.1频域和空域图像增强方法
按照进行处理所在空间的不同,常见的图像增强方法可分为频域处理方法和空域处
理方法两类
\begin{itemize}
\item频域增强方法
  频域方法一般需借助\textbf{傅里叶分析}等变换方法将圈像转换至频率域后再做\textbf{滤波}等处理.
\textbf{同态滤波器}(Homomomorphic Filtering)由Oppenheim等提出,原用于声波的分析和合成,其步骤是将信号转至频域后进行非线性滤波,最后再转回时域.
  后有学者将其用于其他信号及图像的增强.
  方法通过将图像分解为照度和反射两部分分量,同时进行灰度范围的压缩和对比度增强来改进一幅图像的外观.
  通过对数运算,被视作乘性噪声的照度成分可被转化为加性噪声.
  图像中的照度和反射成分本是不可明确分开的,但二者在频域中的位置可大致确定.
  图像中的高频成分往往对应反射分量,而低频分量则对应照度在空间的变化.
  通过压制低频分量,扩大高频分量,即可有效改善图像的视觉效果.
  Polesel等提出了一种非锐化掩模(Unsharp Masking)用于图像对比度增强,方法采用
  \textbf{自适应滤波器}控制锐化路径,对图像中细节区域进行较大的增强,而对平坦区域不增强或进行较小增强.G
  uillon等同样基于自适应滤波器掩模,将非线性低通和高通滤波器结合,对图像进行同时的噪声消除和增强处理.
  陈强等通过调制传递函数(Modulation Transfer Function,MTF)补偿调整图像的光能分配,达到增强图像中细节和边缘的目的,进行遥感图像的复原和增强.
\item空域增强方法
在图像处理中,空域是指由像素点组成的空间,而空域方法则直接作用于图像中的
像素点,可表示为以下的算子形式
g
\begin{equation}
  \label{eq:10}
  g(x,y)=E_H(f(x,y))
\end{equation}
其中,$y(x,y)$和$g(x,y)$分别为增强前后的图像,而$E_H$代表增强操作算子.
如果$E_h$仅定义在每个$(x,y)$像素点上,则玩为一种点操作;
如果玩定义在$(x,y)$的某个邻域上,则玩常称为模板操作.
也有的增强方法在图像梯度场$Vf(x,Y)$上进行操作,也可将其归为空域方法.
空域方法按处理策略的不同,又可分为全局处理的方法和区域自适应处理的方法.
全局处理的方法较为简单,不考虑图像中像素点的空间分布,仅对图像中的像素值进行全局一致的调整;
较复杂的区域自适应方法则基于诸如局部对比度、边缘强度等区域信息,或采用基于偏微分方程或变分模型的描述形式.


较多的图像增强算子玩作用于单幅图像$f(x,y)$,也有的增强算子岛作用于一系列图像
$f_1(x,y),f_2(x,y),\dots,f_n(x,y)$.
简单的作用于多幅图像的增强操作可对数幅图像进行对应点位的算术和逻辑运算,以得到一幅新图像.
我们也可根据应用的具体目的设计更复杂的图像间增强操作方法.

本论文主要讨论空域的图像增强方法.
\end{itemize}
\item 色调映射
  色调映射是在有限动态范围媒介上近似显示高动态范甩图像的一项计算机图形学技术,也是一种增强处理.
我们平常可见的场景中的最大亮度(如晴天的太阳)和最小亮度(如夜空中的星星)的
强度之间的比值可高达$10^9$至$10^{10}$,按照动态范围和输出媒介的动态范围之间的大小关系,场景可分为低动态范甬(LowDynamic Range,LDR)、标准动态范N(StanoardDynamicRange,SDR)和高动态范围(HighDynamicRange,VIII)R)三类.
从SDR场景获得的图像不需进行处理即可由输出媒介直接显示,而从LDR和HDR场景获得的图像分别需进行动态范围托伸和压缩处理.
从本质上来讲.色调映射要解决的问题是进行大幅度的对比度衰减.
将场景亮度变换到可以显示的范围,同时要保持图像细节与颜色等对于表现原始场景所必须的重要的信息

简单的全局色调映射方法采用对数函数、伽玛校正等处理图像像素值,从而对图像动态范围进行全局调整.
Braun等的方法中使用了稍复杂的S形曲线.
即Sigmoidal函数进行灰度值调整,且在函数参数设置时考虑了图像的灰度均值和方差以上介绍的全局方法实现简单、运行速度快,但对于远超过监视器显示范嗣的高动态范同图像往往易导致图像部分区域细节丢失,不能得到好的效果通常,电子设备(如CCD等)以线性方式记录进入镜头的光照强度,这与人眼感知到的场景是不一样的较好的色调映射方法必须考虑人类视觉系统的特性,以使增强结果图像接近人眼直接观察到的场景故更合适的方法是采用区域自适应的方法,如基于Retlnex理论的方法,基于梯度域操作的方法等
\item 对比度增强
  图像对比度有多种定义,一般是指图像中各部分之间灰度级反差的程度.
  对比度增强是一种增强原图中各部分间反差的方法.


  直方图调整是指按照一定的映射对图像的像素值进行调整以增强图像对比度的方法.
  灰度直方图口71是灰度级的函数,描述的是图像中具有该灰度级的像素的个数:其横坐
  标是灰度级,纵坐标是该灰度出现的频率或像素的个数.


  直方图调整映射的形式可以是事先确定的,如采用对数函数、Sigmoidal数的形式等,也可以考虑图像的具体情况进行确定.
  其中,直方图均衡化使像素灰度值分布满足(或近似满足)均匀分布.
  设图像的(离散)直方图为$h(n)$,其可视为未归一化的图像像素值分布概率密度函数.
  将其归一化,并记为$p(n)$,则$c[n]=\sum_{i=0}^n$为累计分布函数(Cumulative Distribution Function,CDF).
  将CDF拉伸至像素值动态范围,则可得到直方图均衡化的映射函数,即
  \begin{equation}
    \label{eq:11}
    T[n]=\left\lfloor 255 \sum_{j=0}^{n} p(j)+0.5\right\rfloor
  \end{equation}
  其中,$\lfloor.\rfloor$ 表示不大于自变量的最大整数.
  这里假设处理的是8位图像,即像素值的取值范
  围为区间$『0,255\rfloor$中的整数.
  以上描述的是离散形式的直方图均衡化,接下来介绍连续形式的表达.
  设像素值为归一化到[o,1】之间的实数,己归一化的直方图(即像素值概率分布密度)为日(s),则直方图均衡化的映射为
  \begin{equation}
    \label{eq:12}
   T(s)=\int_{0}^s H(s) d s
  \end{equation}


  另一种直方图调整方法称为直方图规定化,它可使处理后的直方图分布满足用户给定或期望的任意形式,而不限于均匀分布.
  针对直方图均衡化易对图像产生“过增强”的现象。
  Arici等睁引在直方图均衡化方法的基础上增加了一些约束,实现了根据图像自身特性的自适应直方图规定化.
  他们定义了如下关于调整后直方图h的目标函数,
  \begin{equation}
    \label{eq:13}
    f(\boldsymbol{h})=\left\|\boldsymbol{h}-\boldsymbol{h}_{i}\right\|_{2}^{2}+\lambda\|\boldsymbol{h}-\boldsymbol{u}\|_{2}^{2}+\gamma\|\boldsymbol{D} \boldsymbol{h}\|_{2}^{2}
  \end{equation}
  其中。$h$为目标图像的直方图,$u$为均匀分布的直方图,$h_i$为原始图像的直方图,
  $Dh$表示直方图h的差分,而$||.||$表示2一范数.


以上介绍的直方图调整方法均是全局处理的,没有考虑图像局部的信息,在确定变
换或转移函数时也是基于整个图像的统计量.且灰度调整的映射一旦确定,原图中的水
平线保持不变,各处相同灰度值的像素点在调整后灰度值仍相同.更复杂和灵活的方法
是根据图像局部的梯度、边缘等特性决定局部对比度增强的程度.


\item 机器色感一致
色感一致性(Color Constancy)No411是人类视觉系统的特性,是指在不同的光照等条件
下,人眼总是能感知到物体的“实际”颜色,而几乎不受光照等条件的影响.
这是由于人眼感知到的某点的颜色和亮度并不仅仅取决于该点进入人眼光线的绝对强度,还和其周围的颜色和亮度有关.
人类视觉系统的这一特性是在其漫长的进化过程中获得的.色感一致性己被一系列的生理学实验所证实.
但通常情况下相机等设备成像的过程则简单得多,只是线性地记录进入到镜头的光照的强度,这样所成的图像就会受到光照条件的影响,在特殊光线条件下会产生色偏(Color Cast)的现象.
用机器计算的方法从已成的图像中估计光照,并提取一致色感的过程称为机器一致色感(Machine Color Constancy)方法.

常见的机器一致色感算法基于以下两种假设之一:白片假设和灰色世界假设.白片(White Patch)假设是指任何一个场景中至少有一处物体是白色的,即亮度达到了可能的最高值,图像其他部分物体的颜色可以此为参照求得.
常见的基于自片假设的方法有基于路径的Retinex,基于二维点随机选取的Retinex(RSR)等.
而灰色世界(Gray World)假设则认为在复杂的场景中,颜色的平均值是灰色,即各种颜色分量的平均值趋于相等,且取亮度范围的中间值.
常见的基于灰色世界假设的方法有基于中心/环绕的Retinex、
自动颜色均衡化(ACE)等.
Provenzi等提出的区域对比度驱动的图像增强方法(RACE)则基于白片和灰色世界两种假设的折衷.
一致色感方法虽可用于去除图像的色偏现象.
但若场景事实上不满足算法的假设,机器一致色感方法就会失败.
若我们用MSRCR算法No1(一种中心/环绕Retinex方法,基于灰色世界假设)处理一幅蓝天占据大部分场景的照片,处理结果中蓝天区域的颜色会趋向于灰色,这是由于场景并不满足灰色世界假设的缘故.
\end{itemize}


\subsubsection{多曝光融合}
\label{sec:0203}

文献\cite{李艳梅}提出一种基于梯度信息的多曝光融合(Multi-exposure Fusion)高动态范围图像(High Dynamic Range Image, HDRI)合成方法。
为克服通用多曝光融合增强中权值平均等方法中出现的不考虑信息重要性及邻域像素关系所造成的细节损失和模糊等问题,提出依赖于图像曝光质量评估及梯度域信息进行权值设计,对图像进行融合,实验结果表明增强的图像兼具原始图像在暗区和亮区的相应细节,
图像整体效果符合人类视觉感知特性要求。同时提出一种基于分块去混淆(Ghost Removal)的多曝光融合算法。
在处理动态多曝光图像融合增强过程中,最大的问题是如何解决由于运动所产生的混淆现
象。
本文通过提出基于梯度上升优化处理的自适应分块方法,并结合形态学原理,调整分块大小及动态区域块的权值,最终达到混淆去除的目的。
同时利用 Gaussian中心函数窗口滤波,去除在分块融合过程中引入的块边缘不连续性痕迹。
实验结果表明该方法能有效增强多曝光图像并去除混淆问题。

\subsubsection{全景图像拼接}
\label{sec:0204}
\cite{赵书睿}
(1) 研究了图像几何畸变的校正,包括常用的图像几何变换模型、全景图像投
影模型和摄像头失真的校正。引入一种基于几何模型的图像失真校正方法,快速
有效地实现了摄像头失真导致的图像几何畸变的校正。
(2) 分析了基于频域、基于灰度和基于特征的几类图像配准方法。比较了 Harris
角点,尺度不变特征变换和快速鲁棒性特征,确定了比较有效的特征提取方法。
深入研究了特征点提纯和模型变换参数的估计,实现了图像的配准,对手持相机
拍摄的图像的配准也能取得不错的效果。
(3) 针对动态场景中运动物体导致融合鬼影的问题展开了研究,总结现有几种
\textbf{最佳缝合线}的搜索准则,针对它们存在的问题提出了一种改进的最佳缝合线。改
进的最佳缝合线减小了\textbf{曝光差异}的影响并充分利用了相邻像素点间的相似性,同
时提高了算法的速度。研究了多分辨率融合和泊松融合算法,利用傅里叶变换求
解提高了泊松融合的速度。最后利用最佳缝合线与泊松融合实现了自然无缝的全
景图像。

全景图像拼接方法及系统实现
\begin{itemize}
\item 图像亮度调整
在实际拍摄过程中光照情况完全不变的场景是几乎不存在的,光源的变化,拍摄角度的变化,相机的自动对焦等因素都会导致拍摄的图像间存在曝光差异。
当完成图像的坐标变换后,需要对图像进行一定的处理来减小曝光差异的影响。
图像拼接中最常见的就是对图像进行融合,但是当图像中存在较大的曝光差异时直接融合的效果往往不佳。
这就需要首先根据两幅图像的情况,进行亮度调整来减小图像间的曝光差异。
\item 传统图像融合方法
  \begin{itemize}
  \item 直接平均法
  \item 加权平均法
  \item 距离权重法
  \item 对比度调制法
  \item 图像融合算法分析
  \end{itemize}
\end{itemize}

\cite{张芮}提出了一种新的车牌识别系统的实现方法,使得在各种不同的光照条件下,
通过使用一个具有较大的动态变化范围的传感系统来获取更精确的车牌图像,这个传感
系统是通过对两个不同的\textbf{曝光}条件下的图像进行合成,进而来扩展其动态变化范围的。
为了避免快速行驶车辆所造成的图像模糊,安装了一个相当于多层过滤器的棱形电子束
分裂器,通过调整其入射光与透射光的强度比率来实现图像获取。实验结果表明,提出的
这种系统对于车牌识别是很有效的。

通过合并 ’ 个由不同的 CCD 摄像头在同一时间、不同的曝光条件下所拍摄的两幅图像,那么就可获得一幅具有较大的动态变化范围而无模糊的图像。
这部分主要介绍了该方法,并提出了改进的传感系统的结构。

方法 电子束分裂器将入射光分离成不同强度的反射光和透射光。
分离光线的强度比率$(\lambda_1,\lambda_2)$用光束分裂器内的多层过滤器控制。
不同强度的光线由 2 个CCD 摄像机同时摄入,即传感系统要同时摄入两个不同曝光条件下的图像。

合成两个不同强度的图像后,图像的动态变化范围要比最初产生的图像大得多。
令$\lambda_1>\lambda_2$,
则有下面关系成立:
\begin{equation}
  \label{eq:14}
0 \leq f_2(x,y) \leq f_1(x,y) \leq L_{sat}
\end{equation}
式中 $f_1(x,y),f_2(x,y)$代表通过 CCD1 和 CCD2拍摄的图像上某个像素点坐标$(x,y)$的灰度等级;
$ L_{sat}$ 表示饱和度。
在研究中,来自于摄像机的信号输出数字化为 8 个字节,因而 $ L_{sat}=225$ 。
由$f_1$ 和$f_2$ 具有扩展动态范围的合成图像 $f_{sync}$可通过下面的公式计算出来:

\begin{equation}
  \label{eq:15}
  f_{\mathrm{sync}}(x, y)=\left\{\begin{array}{cl}
f_{1}(x, y), & \text { if } f_{1}(x, y)<L_{\mathrm{sat}} \\
\left(E_{2} / E_{1}\right)^{\gamma} f_{2}(x, y), & \text { if } f_{1}(x, y)=L_{\mathrm{sat}}
\end{array}\right.
\end{equation}

式中, $E_1$ 和 $E_2$是由曝光条件决定的系数,在这里,由于 $\gamma$ 是作为 $gamma$补偿的一个系数,
因此具动态变化的图像,$E_1/E_2$ 的比率应与其中一个摄像机的强度比率
$(\lambda_1:\lambda_2 )$保持一致。
最后,描述了解决密度比率 $(\lambda_1/\lambda_2)$的方法。
传感系统D的动态范围由式~(\ref{eq:16})

计算得出:
\begin{equation}
  \label{eq:16}
D=\left(L_{\mathrm{sat}} / L_{\mathrm{noi}}\right)^{1 / \gamma}\left(E_{1} / E_{2}\right)
\end{equation}

文献 \cite{徐培风}还将BP神经网络应用于自动曝光的图像处理技术中,提出了一种新的
基于图像处理的自动曝光控制算法。
该算法首先将图像分块,将每块予图像的亮度信息作为BP神经网络的输入求出图像的合适曝光量,根据该曝光量确定快门速度和光圈系数,从而有效控制数码相机的曝光。


文献\cite{谢伟}为有效消除引导滤波平滑图像后产生的光晕现象,提出一种新型的融合梯度信息的改进引导滤波算法。
方法该算法借助引导图像的梯度信息来判断图像边缘位置,并结合指数函数框架设计权值来控制不同图像区域内的平滑倍数,使改进后的引导滤波能够自适应地区分和强调边缘,从而避免边缘附近由于过度模糊所引入的光晕现象。
结果与引导滤波算法相比,本文算法能在保边平滑的同时较好地抑制光晕,并在结构相似性(SSIM)评价和峰值信噪比(PSNR)评价中分别取得最高约$30\%$和$15\%$左右的质量提升.

文献\cite{和文娟}分析了 CMOS 图像传感器的模拟噪声类型,主要以时间和空间噪声为分类基础,详细介绍了三维噪声模型的原理。
在此基础上,设计了一个基于三维噪声模型理论的图像传感器时间和空间噪声测量实验。
利用试实验成像系统拍摄一系列均匀背景图像,然后对所拍摄图像进行一系列的分析和处理,得出时间噪声和空间噪声在不同照度下的具体值,并得出其变化趋势。


空间域图像增强
在曝光不足或曝光过度的情况下,图像灰度范围很窄,图像会看起来很不清楚,很多细节看不出来。
灰度变换可以增强原图各部分的对比度,一般采用\textbf{线性变换},
分阶段线性变换和非线性灰度变换三种。
\begin{itemize}
\item
\end{itemize}
\bibliographystyle{plain}
\bibliography{ref} %这里的这个ref就是对文件ref.bib的引用
\end{document}

\message{ !name(main.tex) !offset(-425) }
